\documentclass[10pt,a4paper,oneside]{scrartcl}
\usepackage[latin1]{inputenc}
\usepackage{amsmath}
\usepackage{amsfonts}
\usepackage{amssymb}
\usepackage{makeidx}
\usepackage{graphicx}
\usepackage{booktabs}
\usepackage[
	authordate,
	strict,
	backend=biber
]
{biblatex-chicago}
\usepackage{mathtools}
\author{}
\title{Blockchain}
\date{}
\addbibresource{~/modules/References.bib}
\begin{document}
\maketitle
\paragraph{Notation}: \texttt{blockchain} %This will make an excellent glossary item, but should probably not be a "main paper" module.
\paragraph{Description}: An evolving datanomical record of agreement which requires majority consensus in order to addend. Generally blockchains are so reliable that a supermajority of agreement is attained fairly quickly. Depending on the rules of the particular Blockchain protocol, a Blockchain may be addended several times a minute or as little as once per hour. In either case participants to the network are only rewarded when agreement is reached. 

\end{document}

