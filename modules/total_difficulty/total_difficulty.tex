\documentclass[10pt,a4paper,oneside]{scrartcl}
\usepackage[latin1]{inputenc}
\usepackage{amsmath}
\usepackage{amsfonts}
\usepackage{amssymb}
\usepackage{makeidx}
\usepackage{graphicx}
\usepackage{booktabs}
\usepackage[
	authordate,
	strict,
	backend=biber
]
{biblatex-chicago}
\usepackage{mathtools}
\author{}
\title{Total Difficulty}
\date{}
\addbibresource{~/modules/References.bib}
\begin{document}
\maketitle
\paragraph{Notation}: \texttt{difficulty}
\paragraph{Description}: The \textit{Total Difficulty} of a block is defined recursively by a function which calculates the difficulty of all blocks prior to the header in the present block.
        	\par
                \begin{tabular}{rl}
           \toprule
           \textbf{Pseudocode} & \textbf{Definition} \\
           \midrule
           \texttt{presentstate(total.difficulty)} & Total difficulty of the \texttt{presentstate}. \\
	   \texttt{presentstate(block.parent)}     & This block's \texttt{parent}. \\
           \texttt{presentstate(block.difficulty)} & \textsl{This block's} difficulty. \\
           \bottomrule
                  \end{tabular}
 

\end{document}
