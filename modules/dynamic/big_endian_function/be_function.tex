\documentclass[10pt,a4paper,oneside]{scrartcl}
\usepackage[latin1]{inputenc}
\usepackage{amsmath}
\usepackage{amsfonts}
\usepackage{amssymb}
\usepackage{makeidx}
\usepackage{graphicx}
\usepackage{booktabs}
\usepackage[
	authordate,
	strict,
	backend=biber
]
{biblatex-chicago}
\usepackage{mathtools}
\usepackage{paracol}
\author{}
\title{Big Endian Function}
\date{}
\addbibresource{~/modules/References.bib}
\begin{document}
\maketitle

\begin{paracol}{3}

\paragraph{Notation}: \texttt{big\_endian}
	\paragraph{Description}: This function expands a positive-integer value to a big-endian byte  array of minimal length. \switchcolumn When accompanied by a $\cdot$ operator, it signals sequence concatenation. \switchcolumn The \texttt{big\_endian} function  accompanies RLP serialization and deserialization.
\end{paracol}
\end{document}

