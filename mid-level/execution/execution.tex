\documentclass[10pt,a4paper,oneside]{scrartcl}
\usepackage[latin1]{inputenc}
\usepackage{amsmath}
\usepackage{listings}
\usepackage{amsfonts}
\usepackage{amssymb}
\usepackage{makeidx}
\usepackage{graphicx}
\usepackage{booktabs}
\usepackage[
	authordate,
	strict,
	backend=biber
]
{biblatex-chicago}
\usepackage{mathtools}
\author{}
\title{Execution}
\date{}
\addbibresource{~/modules/References.bib}
\begin{document}
\maketitle
\paragraph{Notation}: \texttt{execution}
\paragraph{Description}: The execution of a transaction defines the state transition function: \texttt{stf}. However, before any transaction can be executed it needs to go through the initial tests of intrinsic validity. 

\section{Intrinsic Validity}

The criteria for intrinsic validity are as follows:

	\begin{itemize}
		\item The transaction follows the rules for \textsl{well-formed RLPs} (recursive length prefixes.)
		\item The \textsl{signature} on the transaction is valid.
		\item The \textsl{nonce} on the transaction is valid, i.e. it is equivalent to the sender account's current nonce.
		\item The \texttt{gas\_limit} is greater than or equal to the \texttt{intrinsic\_gas} used by the transaction.
		\item The sender's account balance contains the cost required in up-front payment.
	\end{itemize}

Accordingly, the post-transactional state of Ethereum is expressed thus: 
\\


\texttt{transaction(post.state) = stf(present.state, transaction)}  
\\


While the amount of gas used in the execution is expressed: \texttt{stf(gas\_used)} and the accrued log items belonging to the transaction are expressed: \texttt{stf(logsbloom, content)(logsbloom, set)}
Information concering the result of a transaction's execution is stored in the transaction receipt \texttt{tx\_receipt}. The set of log events which are created through the execution of the transaction, \texttt{logs\_set} in addition to the bloom filter which contains the actual information from those log events \texttt{logs\_bloom} are located in the transaction receipt. In addition, the post-transaction state \texttt{post\_transaction(state)} and the amount of gas used in the block containing the transaction receipt post(gas\_used) are stored in the transaction receipt. Thusly the transaction receipt is a record of any given \texttt{execution}. \par

The execution of a valid transaction begins with an irrevocable change made to the state: the nonce of the account of the sender is incremented by one and the balance is reduced by part of the up-front cost.

\texttt{Intrinsic gas} is the amount of gas this transaction requires to be paid prior to execution.

The execution of a valid transaction begins with an irrevocable change made to the state: the nonce of the account of the sender is incremented by one, and the balance is reduced by part of the up-front total cost paid by the sender \texttt{valid.transaction = increment.sender(nonce)}. 

The \texttt{original\_transactor} will differ from the \texttt{sender} if the \texttt{message\_call} or \texttt{contract\_creation} comes from a contract account executing code.
\par

After a transaction is executed, there comes a provisional state\footnote{\texttt{post\_execution(provisional.state)}}, Gas used for the execution of individual EVM opcodes prior to their potential addition to the \texttt{world\_state}\footnote{\texttt{productive\_gas}}, and an associated substate \footnote{\texttt{substate\_a}}. 





\end{document}

