\documentclass[10pt,a4paper,oneside]{scrartcl}
\usepackage[latin1]{inputenc}
\usepackage{amsmath}
\usepackage{amsfonts}
\usepackage{amssymb}
\usepackage{makeidx}
\usepackage{graphicx}
\usepackage{booktabs}
\usepackage[
	   style=ieee
	   ]
	   {biblatex}
\usepackage{mathtools}
\author{}
\title{Machine State}
\date{}
\addbibresource{~/modules/References.bib}
\begin{document}
\maketitle
\paragraph{Notation}: \texttt{machine\_state}
\paragraph{Description}: The machine state consists of several elements:
\begin{itemize}
	\item The amount of remaining gas in each transaction is extracted from information contained in the \texttt{machine\_state} 
	\item A simple iterative loop\supercite{Wood2017} with a boolean return value: 
\begin{itemize}

$X$ is thus cycled (recursively here, but implementations are generally expected to use a simple iterative loop) until either $Z$ becomes true indicating that the present state is exceptional and that the machine must be halted and any changes discarded or until $H$ becomes a series (rather than the empty set) indicating that the machine has reached a controlled halt.

\subsubsection{Machine State}
The machine state $\boldsymbol{\mu}$ is defined as the tuple $(g, pc, \mathbf{m}, i, \mathbf{s})$ which are the gas available, the program counter $pc \in \mathbb{P}_{256}$ , the memory contents, the active number of words in memory (counting continuously from position 0), and the stack contents. The memory contents $\boldsymbol{\mu}_\mathbf{m}$ are a series of zeroes of size $2^{256}$.

For the ease of reading, the instruction mnemonics, written in small-caps (\eg \space {\small ADD}), should be interpreted as their numeric equivalents; the full table of instructions and their specifics is given in Appendix \ref{app:vm}.

For the purposes of defining $Z$, $H$ and $O$, we define $w$ as the current operation to be executed:
\begin{equation}\label{eq:currentoperation}
w \equiv \begin{cases} I_\mathbf{b}[\boldsymbol{\mu}_{pc}] & \text{if} \quad \boldsymbol{\mu}_{pc} < \lVert I_\mathbf{b} \rVert \\
\text{\small STOP} & \text{otherwise}
\end{cases}
\end{equation}
\printbibliography
\end{document}

